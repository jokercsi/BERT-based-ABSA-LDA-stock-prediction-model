%%
%% 研究報告用スイッチ
%% [techrep]
%%
%% 欧文表記無しのスイッチ(etitle,eabstractは任意)
%% [noauthor]
%%

%\documentclass[submit,techrep]{ipsj}
\documentclass[submit,techrep,noauthor]{ipsj}

\usepackage{mathtools}
\usepackage[dvips]{graphicx}
\usepackage{latexsym}
\bibliographystyle{ipsjunsrt}

\def\Underline{\setbox0\hbox\bgroup\let\\\endUnderline}
\def\endUnderline{\vphantom{y}\egroup\smash{\underline{\box0}}\\}
\def\|{\verb|}

\begin{document}

\title{Fusion of Sentiment and Asset Price Predictions for Portfolio Optimization}


\author{キム ジビン}{}{} %氏名のところに名前を記入

\begin{abstract}
株ポートフォリオの最適化を紹介する. 近年、感情分析を利用した株の価格予測は金融コミュニティーでは既に注目されている. だが、ポートフォリオ選択方法に関する研究は未だに少ない. 
本論文では、感情分析を利用したポートフォリオ選択方法の改善を目的とする. 具体的には、Semantic Attentionモデルを利用して資産に対する感情を予測する.そして、価格予測とmean-varience戦略のためLSTMモデルを活用して最適なポートフォリオを選ぶ.我々の感情ポートフォリオ戦略は非感情ポートフォリオより収入率を改善した.だが、資産運用の安全性の面では、旧来のポートフォリオより優れる結果を出すごとができなかった.
\end{abstract}


\maketitle
% ここから本文

%1
\section{はじめに}
ポートフォリオ最適化は本質的に複雑な金融計算問題である.ポートフォリオ選択とは、最も良い富の配分を探すことである. 研究者らは様々な方法でこの問題を解決しようとした.その中で一番認められている技術は\cite{first}である.近年では、投資家の感情を特徴として予測するモデルが人気になっている. \cite{second}多くの論文が価格予測や感情分析、ポートフォリオ最適化の技術を融合することで良い結果を出している. 今回の論文では、感情認識ポートフォリオ選択の強化を調査する.はじめに、感情分析モデルVADERが偏りを持つ結果を出すことから紹介する.この結果は、VADERモデルを改善した感情モデルSemantic Attention model を使う動機となる.感情認識戦略は上昇トレンドと下降トレンドの市場で従来のポートフォリオ配分戦略を上回る. ただし、我々の戦略は従来のアプローチよりも堅牢性が低く、リスクへのエクスポージャーが増加した.この研究では、感情を表すさまざまな指標を調査し、テキストデータの誤った分類による偏ったラベルを処理する方法を実装する.研究の結果では、2つのポートフォリオ設計を比較する.1つは従来のmean-varience ポートフォリオ最適化によるもので、もう1つは資本の最適な配分を予測するLSTMフレームワークによるものである. この調査は、感情を意識したポートフォリオ戦略が、従来のポートフォリオ選択方法よりも高いリターンを生み出すことを示している.そして、分析的に価格と投資家の感情が実際に相関していることを詳しく説明する. また、これらの結果から感情分析ツールにかなりの偏りがあることも示す.


%2
\section{提案手法}
\subsection{価格データ}
ポートフォリオはDow Jones Industrial Average (DJIA) から5つの企業で構成.価格データはYahoo Financeから収集する.毎日の終値調整後の価格データは、2001年1月1日から2018年12月31日までの範囲です.価格データは、70%の訓練データ、20%のテストデータ、10%の検証データに分割される.訓練セットの範囲は2001年1月1日から2013年12月11日まで、検証セットは2013年12月12日から2015年5月20日,テストセットは2015年5月21日から2018年12月31日まで.
\subsection{感情データ}
Google NewsとTwitterから感情データを取得する. Selenium WebDriverを使用して5社に関連するニュース記事をスクレイピングする. SnscrapeはTwitterのツイートのスクレーパーとして使用される.感情データの訓練、テスト、検証の分割は価格データで行われた分割と同様に分割される.また、7日間感情データを一つのウィンドウに設定する.
\subsection{ラベリング}
他の分類問題とは異なり、本実験のデータにはラベルが付けられていない.つまり、記事やツイートのデータにはポジティブまたはネガティブな感情を認識できない.このような場合、ほとんどの研究では感情分析ツールVADERを使用する.VADERは俗語、顔文字、絵文字、収縮、およびその他の自然言語属性を解釈できる. VADERは、語彙の特徴を感情の強さにマッピングする辞書を使用して、感情をポジティブまたはネガティブとして定量化する. 

%3
\section{実験}
\subsection{感情と収入率の相関関係}
収益と感情の相関関係を統計的に表すことで、資産に対する市場の動きを捉えることが可能か確かめる.結果的に収益より価格のデータを利用したLSTMモデルを使用することに決定.Granger’s Causality テストを利用して時差がある感情時系列が収益を予測可能かを検証する.
\begin{table}[htb] % 表の宣言.図○のときは figure,表○のときは table と覚えましょう
\caption{Performance metrics} % 表のタイトル,図と同じ使い方
\label{release} % 参照の際に使う表のラベル,図や数式と同じ使い方
\begin{center}
\begin{tabular}{| l | c | c  | c | c |} \hline % [|] は縦罫線,\hline は横罫線
Asset & mean & max & median & ratio  \\ \hline % 上の項目数に合わせて,項目を&で区切ると縦
3M & 0.12 & 0.08 & 0.14 & 0.09 \\ 
Microsoft &0.18 & 0.08& 0.12 &  0.13 \\
Disney  &0.17 &   0.08 & 0.16 &  0.11\\
Nike   & 0.05& 0.07 & 0.06 &  0.0.1\\
Walmart  &0.14 &0.01 & 0.14 &  0.14\\ \hline
\end{tabular}
\end{center}
\end{table}

\subsection{モデル}
本研究ではテキストから感情を予測するため Semantic Attention Model \cite{third}を使用する.これは重要度の高い単語をフォーカスする特徴を持つ.また、価格予測は感情認識LSTMモデルを使用する.LSTMのインプットは株価、いいね数、リツイート数、コメント数、ポジティブ&ネガティブ感情で構成される.そして予測した価格をもとにmean-varianceで最適なポートフォリオを選ぶ.これは収益を最大(1)、リスクを最小(2)を目標としている.モンテカルロ法で50,000個のポートフォリオ生成して、効率的フロンティアのポートフォリオを選択する.


\begin{equation}
  max\sum_{i=1}^N x_i \mu_i
\end{equation}

\begin{equation}
  	s.t\left\{
		\begin{alignedat}{2} 
		& \sum_{i=1}^N x_i = 1\\
		& x_i  \ge 0, \forall i  = 1, \dots , n
		\end{alignedat} 
	\right.
\end{equation}

\subsection{ベンチマーク}
Semantic Attention model,LSTM価格予測,mean-varienceを融合したモデルをこれからLSTM+Sと呼ぶことにする.LSTM+Sの性能を評価するため従来のポートフォリオと比較をする.感情認識なしのLSTM,ポートフォリオの割合を変えないBuy and Holdポートフォリオ,t時間の間隔で毎回一定の割合を変えるRebalancingポートフォリオ,すべての資産を一番収入率が良い株種目に割り当てるBestStockポートフォリオがある.


%4
\section{結果}
平均的な収入率を得るBuy and Holdポートフォリオをベンチマークにしてポートフォリオ最適化を行う.表2の結果で従来の手法よりLSTMを使用するポートフォリオが性能が良いことがわかる.さらにLSTM+Sモデルは最も良い収入率を見せている.表3では低下傾向の市場での精度を比較している.感情なしLSTMモデルがLSTM+Sより良い収入率を得ている.この理由はネガティブラベルを十分に持っておらず、ポジティブラベルに偏りがあるためだと考えられる.



\begin{table}[htb] % 表の宣言.図○のときは figure,表○のときは table と覚えましょう
\caption{Performance metrics} % 表のタイトル,図と同じ使い方
\label{release} % 参照の際に使う表のラベル,図や数式と同じ使い方
\begin{center}
\begin{tabular}{| l | c | c  | c | c |} \hline % [|] は縦罫線,\hline は横罫線
Models & Capital & fAPV &BV & SR  \\ \hline % 上の項目数に合わせて,項目を&で区切ると縦
Buy and Hold & 14856.68 & 1.49 & 1.00 &  1.00 \\ 
Best Stock &17871.20 &1.79& 1.20 &  4.17 \\
Rebalancing  &14781.47 &  1.48 & 0.99 &  0.91\\
LSTM  & 16966.97 &1.70 & 1.14 &  3.26\\
LSTM + S  &18047.80 &1.80 & 1.21 &  1.17\\ \hline
\end{tabular}
\end{center}
\end{table}

\begin{table}[htb] % 表の宣言.図○のときは figure,表○のときは table と覚えましょう
\caption{Performance metrics for a down market} % 表のタイトル,図と同じ使い方
\label{release} % 参照の際に使う表のラベル,図や数式と同じ使い方
\begin{center}
\begin{tabular}{| l | c | c  | c | c |} \hline % [|] は縦罫線,\hline は横罫線
Models & Capital & fAPV &BV & SR  \\ \hline % 上の項目数に合わせて,項目を&で区切ると縦
Buy and Hold & 9766.90 & 0.98 & 1.00 &  5.27 \\ 
Best Stock &9972.24 &0.99& 1.02 & 7.68 \\
Rebalancing  &9809.24 &  0.98 & 1.00&  5.39\\
LSTM  & 10978.207  &1.10& 1.12 & -14.57\\
LSTM + S  & 10702.91 &1.07 & 1.10 & -2.01\\ \hline
\end{tabular}
\end{center}
\end{table}

%5
\section{結論}
本実験はポートフォリオの最適化を目標としている.今回VADERモデルの感情ラベルは偏りを持つことがわかった.また、感情認識ポートフォリオは従来より収入率が良く、特に低下傾向の市場に多くの収入を得た.その反面、リスク率は従来のポートフォリオより改善できなかった.実験の改善点としては、出版社によって既にラベリングされているデータを使用することが感情分析の制度面で良いと考えられる.

\bibliography{bibsample}

\end{document}
